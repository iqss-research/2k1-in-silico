% !TEX program = pdflatex
\documentclass[12pt]{article}
\usepackage[reqno]{amsmath}
\usepackage{amssymb,amsthm,graphicx,verbatim,url,verbatim,longtable,vmargin,accents,bbm,times,subfig,dcolumn,booktabs,setspace,soul,latexsym,wasysym,titling,enumitem,longtable,booktabs,xr}
\externaldocument{acc-supp}
\allowdisplaybreaks
% \usepackage{showkeys}

\usepackage[export]{adjustbox}
\usepackage[dvipsnames,usenames]{xcolor}
\definecolor{spot}{rgb}{0.6,0,0}

\usepackage{tikz}\usetikzlibrary{tikzmark,arrows,calc,arrows.meta}
\usepackage[T1]{fontenc}\usepackage[encapsulated]{CJK}\usepackage[utf8]{inputenc}

\usepackage[natbib=true,uniquename=false,minbibnames=1,maxbibnames=99,maxcitenames=1,maxcitenames=3,backend=biber,ibidtracker=false,style=authoryear]{biblatex}
\addbibresource[location=remote]{https://raw.githubusercontent.com/iqss-research/gkbibtex/master/gk.bib}
\addbibresource[location=remote]{https://raw.githubusercontent.com/iqss-research/gkbibtex/master/gkpubs.bib}
\setcounter{biburllcpenalty}{7000}\setcounter{biburlucpenalty}{8000}

\usepackage[all]{xy}
\setpapersize{USletter} \topmargin=0in
\newcolumntype{.}{D{.}{.}{-1}}\newcolumntype{d}[1]{D{.}{.}{#1}}
\graphicspath{{./figs/}}
\renewcommand{\topfraction}{0.85} \renewcommand{\textfraction}{0.1}
\renewcommand{\floatpagefraction}{0.75} % keep < \topfraction
\newcommand{\cntext}[1]{\begin{CJK}{UTF8}{gbsn}#1\end{CJK}}
\newcommand{\btVFill}{\vskip0pt plus 1filll}
\usepackage[titletoc,title]{appendix}
\newtheorem{proposition}{Proposition}
\DeclareMathOperator*{\argmax}{arg\,max}
\DeclareMathOperator*{\argmin}{arg\,min}
\newcommand{\mean}{\operatornamewithlimits{mean}}  
\newcommand{\Cov}{\text{Cov}}
\theoremstyle{definition}
\newcommand{\blind}{0} % 1=blind, 0=not blind
\newcommand{\titl}{Statistical Intuition Without Coding (or Teachers)}
\newcommand{\authr}{Natalie Ayers, Gary King, Zagreb Mukerjee, Dominic Skinnion}

\if1\blind
\title{\titl}
\renewcommand{\authr}{}
\fi
\usepackage[pdftex, bookmarksopen=true, bookmarksnumbered=true,
  pdfstartview=FitH, breaklinks=true, urlbordercolor={0 1 0},
  citebordercolor={0 0 1}, colorlinks=true, citecolor=spot, 
  linkcolor=spot, urlcolor=spot, pdfauthor={\authr},
  pdftitle={\titl}]{hyperref}

\if0\blind
\title{\titl\thanks{Our thanks to ... for helpful comments.}}
%
\author{Natalie Ayers\thanks{}\and Gary King\thanks{Albert J.\ Weatherhead
III University Professor, Institute for Quantitative Social
Science, Harvard University; GaryKing.org, King@Harvard.edu.}\and Zagreb Mukerjee\thanks{} \and Dominic Skinnion\thanks{}}
\fi

\begin{document}
\maketitle\thispagestyle{empty}\setcounter{page}{0}
\btVFill
\vspace{-2\baselineskip}
\begin{abstract}
  \noindent Two features of political methodology make teaching and learning especially difficult: (1) Understanding each part of its sophisticated theories (of inference and mathematical statistics) is difficult without first understanding all the other parts; and (2) these abstract theories are usually taught simultaneously with the practical details of a statistical programming language (such as R).  Learning divergent skills simultaneously is harder, but learning them sequentially can be demotivating for aspiring political scientists eager to see the connection with substantive research.  We address both problems through a new type of automated teaching tool, so that students can see the big theoretical picture and all its separate parts at the same time, and also do not have to simultaneously learn coding or lose motivation. This tool, which we make available via one click in a web browser, is also designed to work without instructor supervision.
  \\
  \newline
\noindent Words: [a count] 
\end{abstract}
\btVFill
\clearpage
% \renewcommand{\contentsname}{Contents (page to be removed before publication)}
% \setcounter{tocdepth}{8}\tableofcontents\clearpage
\baselineskip=1.57\baselineskip

\section{Introduction}\label{s:intro}

Most new political science Ph.D.\ students have long since branched off from math and physics and are excited to be able to focus on their substantive interests in government and politics. Yet, upon arrival, they are sometimes surprised to learn their first class is in quantitative political methodology, and they now need to master some highly sophisticated statistical concepts. Since ``deferral of gratification'' pretty much defines the graduate school experience, most dutifully go along. But then they arrive in class, expecting to learn these statistical concepts and are told that they must first learn a statistical programming language --- in order that they can learn the sophisticated statistical concepts, in order that they can begin to study what they came to graduate school for in the first place.

Moreover, political methodology classes are unusually difficult because they require learning both a hands-on statistical programming language --- leaving you to correct an obscure error on line 57 --- and sophisticated theoretical concepts in statistical inference (such as why the math behind probability distributions can serve as data generation processes, given chosen parameters, and also turned around to estimate parameters from observed data). Not only do these divergent tasks not go well together as learning exercises, but learning any one part of either one is difficult without learning all the other parts.

as they are trying to learn, usually for the first time, probability theory, the likelihood theory of inference, and statistical simulation for computing quantities of interest

In this paper, we attempt to remove a level of learning complexity from this stack of requirements and make it possible to learn the necessary sophisticated statistical concepts without having to learn how to program R.  Of course, learning a statistical programming language will eventually prove useful for practical data analysis, but 

making it easier to motivate students and quicker for them to begin scholarly research.  


- flexible, ability to use many models; adjust all parameters and X var values
- seeing graphics respond instantly and dynamically to input changes
- a wide array of models
- used for probability, statistical inference, and presenting results
- full math, presented dynamically for ev model.  sw rarely has it. when it does, even the formatting is bad or incomplete.
- in context tips and help, with big picture and details
- a tutorial that explains the interface
- enables user to make guestimates and see consequences before choosing to optimize




\singlespace
\printbibliography
\end{document}

