% !TEX program = pdflatex
\documentclass[12pt]{article}
\usepackage[reqno]{amsmath}
\usepackage{amssymb,amsthm,graphicx,verbatim,url,verbatim,longtable,vmargin,accents,bbm,times,subfig,dcolumn,booktabs,setspace,soul,latexsym,wasysym,titling,enumitem,longtable,booktabs,xr}
\externaldocument{acc-supp}
\allowdisplaybreaks
% \usepackage{showkeys}

\usepackage[export]{adjustbox}
\usepackage[dvipsnames,usenames]{xcolor}
\definecolor{spot}{rgb}{0.6,0,0}

\usepackage{tikz}\usetikzlibrary{tikzmark,arrows,calc,arrows.meta}
\usepackage[T1]{fontenc}\usepackage[encapsulated]{CJK}\usepackage[utf8]{inputenc}

\usepackage[natbib=true,uniquename=false,minbibnames=1,maxbibnames=99,maxcitenames=1,maxcitenames=3,backend=biber,ibidtracker=false,style=authoryear]{biblatex}
\addbibresource[location=remote]{https://raw.githubusercontent.com/iqss-research/gkbibtex/master/gk.bib}
\addbibresource[location=remote]{https://raw.githubusercontent.com/iqss-research/gkbibtex/master/gkpubs.bib}
\setcounter{biburllcpenalty}{7000}\setcounter{biburlucpenalty}{8000}

\usepackage[all]{xy}
\setpapersize{USletter} \topmargin=0in
\newcolumntype{.}{D{.}{.}{-1}}\newcolumntype{d}[1]{D{.}{.}{#1}}
\graphicspath{{./figs/}}
\renewcommand{\topfraction}{0.85} \renewcommand{\textfraction}{0.1}
\renewcommand{\floatpagefraction}{0.75} % keep < \topfraction
\newcommand{\cntext}[1]{\begin{CJK}{UTF8}{gbsn}#1\end{CJK}}
\newcommand{\btVFill}{\vskip0pt plus 1filll}
\usepackage[titletoc,title]{appendix}
\newtheorem{proposition}{Proposition}
\DeclareMathOperator*{\argmax}{arg\,max}
\DeclareMathOperator*{\argmin}{arg\,min}
\newcommand{\mean}{\operatornamewithlimits{mean}}  
\newcommand{\Cov}{\text{Cov}}
\theoremstyle{definition}
\newcommand{\blind}{0} % 1=blind, 0=not blind
\newcommand{\titl}{Learning Complicated Statistics Without Coding (or Teachers)}
\newcommand{\authr}{Natalie Ayers, Gary King, Zagreb Mukerjee, Dominic Skinnion}

\if1\blind
\title{\titl}
\renewcommand{\authr}{}
\fi
\usepackage[pdftex, bookmarksopen=true, bookmarksnumbered=true,
  pdfstartview=FitH, breaklinks=true, urlbordercolor={0 1 0},
  citebordercolor={0 0 1}, colorlinks=true, citecolor=spot, 
  linkcolor=spot, urlcolor=spot, pdfauthor={\authr},
  pdftitle={\titl}]{hyperref}

\if0\blind
\title{\titl\thanks{Our thanks to ... for many helpful comments.}}
%
\author{Natalie Ayers\thanks{}\and Gary King\thanks{Albert J.\ Weatherhead
III University Professor, Institute for Quantitative Social
Science, Harvard University; GaryKing.org, King@Harvard.edu.}\and Zagreb Mukerjee\thanks{} \and Dominic Skinnion\thanks{}}
\fi

\begin{document}
\maketitle\thispagestyle{empty}\setcounter{page}{0}
\btVFill
\vspace{-2\baselineskip}
\begin{abstract}
  \noindent Most political methodology instructors teach highly sophisticated theoretical concepts in inference and mathematical statistics while simultaneously insisting that students learn the hands-on, practical details of a statistical programming language, such as R. The reasoning behind needing these two almost opposite skills is sound --- theory gives us understanding and the programming language helps build intuition and make the theory practical --- the two can conflict while learning. While
  \\
  \newline
\noindent Words: [a count] 
\end{abstract}
\btVFill
\clearpage
% \renewcommand{\contentsname}{Contents (page to be removed before publication)}
% \setcounter{tocdepth}{8}\tableofcontents\clearpage
\baselineskip=1.57\baselineskip

\section{Introduction}\label{s:intro}

We study \emph{electoral accountability}, a universally recognized criterion for healthy democracies. Although many important versions of this concept have been studied \citep{CanBraCog02, AnsJon10, HirSny12, NyhMcgSid12, TauWar18, FraHer2018, AnsKur22, FouHal22, IarLopMei22}, its most agreed upon essential component --- the ability of citizens to hold legislators accountable via threat of electoral defeat --- has rarely been directly quantified \citep{PrzAlv00}. This concept is especially valuable because it leaves to citizens, rather than researchers, the choice of how far and in what way the behavior of elected representatives may acceptably deviate from public opinion.


\singlespace
\printbibliography
\end{document}

